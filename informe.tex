%% LyX 2.0.2 created this file.  For more info, see http://www.lyx.org/.
%% Do not edit unless you really know what you are doing.
\documentclass[12pt,oneside,english,spanish]{book}
\usepackage[T1]{fontenc}
\usepackage[latin9]{inputenc}
\usepackage[letterpaper]{geometry}
\geometry{verbose,tmargin=2cm,bmargin=2cm,lmargin=3cm,rmargin=2cm}
\usepackage{fancyhdr}
\pagestyle{fancy}
\setcounter{secnumdepth}{3}
\usepackage{babel}
\addto\shorthandsspanish{\spanishdeactivate{~<>}}

\usepackage{array}
\usepackage{longtable}
\usepackage{float}
\usepackage{pdfpages}
\usepackage{graphicx}
\usepackage{setspace}
\onehalfspacing
\usepackage[unicode=true,
 bookmarks=true,bookmarksnumbered=false,bookmarksopen=false,
 breaklinks=false,pdfborder={0 0 1},backref=section,colorlinks=false]
 {hyperref}
\hypersetup{pdftitle={Sistema de Atenci�n de Consultas M�dicas},
 pdfauthor={Miguel Miguel Ambrosio G�mez},
 pdfkeywords={USB, Pasant�a Larga, Venetecnolog�a, Consultas M�dicas,  Agile UP, MVC, Cliente-Servidor, Aplicaci�n Web, CakePHP, PHP, jQuery.}}
\usepackage{breakurl}

\makeatletter

%%%%%%%%%%%%%%%%%%%%%%%%%%%%%% LyX specific LaTeX commands.
\newcommand{\noun}[1]{\textsc{#1}}
%% Because html converters don't know tabularnewline
\providecommand{\tabularnewline}{\\}
%% A simple dot to overcome graphicx limitations
\newcommand{\lyxdot}{.}


%%%%%%%%%%%%%%%%%%%%%%%%%%%%%% User specified LaTeX commands.
 \usepackage{titlesec}
\newcommand*{\justifyheading}{\raggedright}

\titleformat{\chapter}[display]
 {\normalfont\Large\bfseries\centering}
 {\chaptertitlename\ \thechapter}{0pt}{\large}
\titlespacing*{\chapter}{0pt}{-30pt}{20pt}

\titleformat{\section}
 {\normalfont\normalsize\bfseries\justifyheading}
 {\thesection}{1em}{}

\titleformat{\subsection}
 {\normalfont\normalsize\bfseries\justifyheading}
 {\thesubsection}{1em}{}

\titleformat{\subsubsection}
 {\normalfont\normalsize\bfseries\justifyheading}
 {\thesubsubsection}{1em}{}

\usepackage{pdfpages}
\pretolerance=2000
\tolerance=3000

\makeatother

\begin{document}
\pagestyle{empty}

\begin{center}
\includegraphics[scale=0.5]{cebolla}
\par\end{center}

\noindent \begin{center}
\textbf{\noun{\large UNIVERSIDAD SIM�N BOL�VAR}}\textbf{\noun{}}\\
\textbf{\noun{DECANATO DE ESTUDIOS PROFESIONALES}}\\
\textbf{\noun{COORDINACI�N DE INGENIER�A DE LA COMPUTACI�N}}\noun{}\\
\bigskip{}
\bigskip{}
\bigskip{}
\bigskip{}
\bigskip{}
\bigskip{}
\bigskip{}
\bigskip{}
\bigskip{}
\bigskip{}
\bigskip{}
\bigskip{}

\par\end{center}

\noindent \begin{center}
\textbf{\large SISTEMA DE ATENCI�N DE CONSULTAS M�DICAS}
\par\end{center}{\large \par}

\bigskip{}
\bigskip{}
\bigskip{}
\bigskip{}
\bigskip{}
\bigskip{}


\begin{center}
Por:\\
Miguel Miguel Ambrosio G�mez
\par\end{center}

\bigskip{}
\bigskip{}


\bigskip{}
\bigskip{}
\bigskip{}
\bigskip{}
\bigskip{}
\bigskip{}


\begin{center}
\textbf{\large INFORME DE PASANT�A}
\par\end{center}{\large \par}

\begin{center}
Presentado ante la Ilustre Universidad Sim�n Bol�var\\
como requisito parcial para optar por el t�tulo de\\
Ingeniero en Computaci�n
\par\end{center}

\bigskip{}
\bigskip{}
\bigskip{}
\bigskip{}
\bigskip{}
\bigskip{}
\bigskip{}
\bigskip{}


\vfill{}


\begin{center}
\textbf{Sartenejas, Enero de 2012}
\par\end{center}


\cleardoublepage{}

\include{portada}

\cleardoublepage{}

\begin{center}
\includegraphics[scale=0.5]{cebolla}
\par\end{center}

\begin{center}
\textbf{\noun{\large UNIVERSIDAD SIM�N BOL�VAR}}\textbf{\noun{}}\\
\textbf{\noun{DECANATO DE ESTUDIOS PROFESIONALES}}\\
\textbf{\noun{COORDINACI�N DE INGENIER�A DE LA COMPUTACI�N}}\\
\bigskip{}
\bigskip{}
\bigskip{}
\bigskip{}
\bigskip{}
\bigskip{}

\par\end{center}

\noindent \begin{center}
\textbf{ACTA FINAL PASANT�A LARGA}
\par\end{center}

\bigskip{}
\bigskip{}


\noindent \begin{center}
\textbf{\large SISTEMA DE ATENCI�N DE CONSULTAS M�DICAS}
\par\end{center}{\large \par}

\bigskip{}


\bigskip{}


\bigskip{}


\begin{center}
Presentado por:\\
Miguel Miguel Ambrosio G�mez
\par\end{center}

\bigskip{}
\bigskip{}


\begin{center}
Esta Pasant�a Larga ha sido aprobada por el siguiente jurado examinador:
\par\end{center}

\bigskip{}


\bigskip{}
\bigskip{}


\begin{center}
\line(1,0){200}\\
Prof. Carlos Alberto P�rez\\
\textbf{Jurado}
\par\end{center}

\bigskip{}
\bigskip{}


\begin{center}
\line(1,0){200}\\
Prof. Edumilis M�ndez\\
\textbf{Tutor Acad�mico} 
\par\end{center}

\bigskip{}
\bigskip{}


\begin{center}
\line(1,0){200}\\
Ing. Miguel �ngel Sucre\\
\textbf{Tutor Industrial} 
\par\end{center}

\bigskip{}
\bigskip{}


\vfill{}


\begin{center}
\textbf{Sartenejas, 25/01/2012}
\par\end{center}




\cleardoublepage{}

\pagestyle{fancyplain}

\fancypagestyle{plain}{ % 
\fancyhf{} % remove everything 
\cfoot{\thepage}
\renewcommand{\headrulewidth}{0pt} % remove lines as well 
\renewcommand{\footrulewidth}{0pt}}


\pagestyle{plain}

\renewcommand{\thepage}{\Roman{page}}

\setcounter{page}{4} 

\begin{center}
\textbf{\large Resumen}
\par\end{center}{\large \par}

\vspace{5mm}


\begin{onehalfspace}
El presente proyecto de pasant�a consiste en el dise�o e implementaci�n
de una aplicaci�n m�dica dise�ada para atender consultas m�dicas a
trav�s de un navegador Web. La empresa VENETECNOLOG�A C.A. (filial
del GRUPO VENEMERGENCIA) tiene como objetivo la creaci�n de esta aplicacion
cuyo fin es permitirle a un m�dico poder atender a sus pacientes,
almacenando la informaci�n de cada consulta m�dica, para as�, mantener
un historial completo del paciente. Este historial permitir�a enriquecer
la informaci�n de los pacientes, lo que puede conllevar al alcance
de diagn�sticos m�s precisos, reinvertir los tiempos de consulta utilizados
para recopilar informaci�n que ya se posee, adem�s mejorar y agilizar
los procedimientos que involucra el desarrollo de una consulta.

La idea es construir un conjunto de formularios, que de manera intuitiva
permitan el desarrollo r�pido y efectivo de una consulta m�dica. Tambi�n,
permitir acceder al m�dico a informaci�n de consultas previas En base
a esta informaci�n el m�dico debe poder registrar consultas, recetar
prescripciones, solicitar citas de laboratorio o un traslado de ambulancia
hasta la ubicaci�n del paciente. La aplicaci�n debe mejorar el uso
del tiempo al momento de atender un paciente.
\end{onehalfspace}

La aplicaci�n tambi�n debe automatizar la generaci�n de reportes e
informes m�dicos, en base a la informaci�n recopilada en cada consulta,
tambi�n, debe presentar una vista de reportes estad�sticos, para obtener
informaci�n acerca de las patolog�as recurrentes diagnosticadas entre
los pacientes que han sido consultados. Esta aplicaci�n debe trabajar
en conjunto con el sistema de telemedicina y gesti�n de consultas
m�dicas v�a celular (ContactaMed), que se encuentra en desarrollo,
y el sistema ya existente de citas de laboratorio (VeneLab). 

\begin{onehalfspace}
La metodolog�a utilizada para la implementaci�n de este proyecto es
Agile UP, una metodolog�a de tipo �gil, con un buen balance entre
la planificaci�n del proyecto y tiempos de desarrollo del software.\newpage{}\end{onehalfspace}



\cleardoublepage{}

\renewcommand{\thepage}{\Roman{page}}

\setcounter{page}{5} 

\vspace*{4cm}
 

\begin{center}
{\large DEDICATORIA}
\par\end{center}{\large \par}

\vspace*{4cm}


\noindent \begin{flushright}
\textit{A mis padres,}\\
\textit{sin su esfuerzo y apoyo incondicional}\\
\textit{nunca habr�a alcanzado mis metas,}\\
\textit{son el tesoro m�s grande que tengo.}\\
\textit{A mis hermanas,}\\
\textit{nada reemplaza su amistad y cari�o.}\\

\par\end{flushright}

\vfill{}


{\large \newpage{}}


\cleardoublepage{}

\selectlanguage{english}%
\pagestyle{empty}

\selectlanguage{spanish}%
\vspace*{3cm}


\textbf{\LARGE Agradecimientos}{\LARGE \par}

~

\textsl{Muchas gracias a Venetecnolog�a C.A. y al Grupo Venemergencia,
por brindarme la oportunidad de participar activamente en el desarrollo
de este proyecto, en especial a los Dres. Andr�s Gonz�lez y Luis Vel�squez,
quienes siempre estuvieron dispuestos a aportar ideas y prestar toda
la ayuda necesaria. Tambi�n los compa�eros de trabajo Yvonne, Alberto,
Viviana, Aldo, �scar, Ronny y el resto de personas que forman y formaron
parte de la familia Venemergencia. Tambi�n a mis grandes amigos y
compa�eros de oficina, en Venetecnolog�a, Daniel Ram�rez y David Moreno,
con quienes form� un excelente y ameno equipo de trabajo, y quienes
mediante sus consejos y apoyo me ense�aron muchas cosas y me ayudaron
a culminar mis objetivos con �xito.}

\textsl{Agradezco a mi tutor industrial, el Ing. Miguel Sucre, y al
Ing. Ricardo Blanch, por todo el apoyo, experiencia y tiempo brindados
durante la pasant�a. Sus ideas y puntos de vista fueron muy importantes
en cada etapa del proyecto.}

\textsl{Brindo un agradecimiento especial a mi tutora acad�mica, la
Prof. Edumilis M�ndez, quien dedic� parte de su valioso tiempo a la
revisi�n y correcci�n de este libro, adem�s de proporcionar una asesor�a
efectiva que fue significativa para el desarrollo de este proyecto,
y aprovecho para agradecer a los dem�s profesores, ayudantes acad�micos
y preparadores de la universidad, que formaron parte de mi proceso
educativo. Muchas gracias por su dedicaci�n y por todas sus ense�anzas.}

\textsl{A mis amigos y compa�eros de carrera Luis Alejandro, Manuel,
Jorge, Sabrina, El�, Christian, Miguel, Ricardo, Jhonathan, Krisvely,
Daniela, Fabiana, Emily, y muchos m�s que compartieron conmigo este
camino de logros y sacrificios llamado universidad, especialmente
a Fernando y David, quienes brindaron una excelente gu�a para la redacci�n
de este libro.}

\textsl{Muchas gracias a mis padres, Enrico y Liana, quienes con su
cari�o, trabajo, esfuerzos y dedicaci�n ayudaron a hacer de m� la
persona que soy. Nada en el mundo puede pagar lo que ustedes me han
dado, los amo. A mis hermanas, que siempre estuvieron y est�n ah�,
dispuestas a apoyarme en todo.}

\textsl{Y a Dios, por estar siempre presente, cuidarme y darme fuerzas
para todo. }

\textsl{A todos ustedes, infinitas gracias.}


\cleardoublepage{}

\pagestyle{fancy}

\addcontentsline{toc}{chapter}{�ndice general}\tableofcontents{}

\cleardoublepage{}

\selectlanguage{english}%
\addcontentsline{toc}{chapter}{�ndice de cuadros}

\selectlanguage{spanish}%
\listoftables


\cleardoublepage{}

\addcontentsline{toc}{chapter}{�ndice de figuras}

\listoffigures


\cleardoublepage{}

\pagestyle{fancy}


\chapter*{{\large Lista de S�mbolos y Abreviaturas}}

\renewcommand{\thepage}{\Roman{page}}

\selectlanguage{english}%
\addcontentsline{toc}{chapter}{Lista de S�mbolos y Abreviaturas}

\selectlanguage{spanish}%
\vspace{5mm}


\begin{onehalfspace}
\begin{center}
\begin{tabular}{llll>{\raggedright}p{11cm}}
ACL &  &  &  & Access Control Lists\tabularnewline
API &  &  &  & Application Programming Interface\tabularnewline
Agile UP &  &  &  & Agile Unified Process\tabularnewline
AJAX &  &  &  & Asynchronous Javascript and XML\tabularnewline
CU &  &  &  & Casos de Uso\tabularnewline
GPL &  &  &  & General Public License\tabularnewline
HTML &  &  &  & HyperText Markup Languaje\tabularnewline
IDE &  &  &  & Integrated Development Enviroment\tabularnewline
JSON &  &  &  & JavaScript Object Notation\tabularnewline
LOPCYMAT &  &  &  & Ley Org�nica para la Prevenci�n, Condiciones y Medio Ambiente de Trabajo\tabularnewline
PHP &  &  &  & PHP: Hypertext Preprocessor\tabularnewline
RUP &  &  &  & Rational Unified Process\tabularnewline
SQL &  &  &  & Structured Query Language\tabularnewline
XAMPP &  &  &  & X (cualquier sistema operativo) Apache, MySQL, PHP, Perl\tabularnewline
XML &  &  &  & eXtensible Markup Language\tabularnewline
\end{tabular}
\par\end{center}
\end{onehalfspace}

\pagebreak{}


\cleardoublepage{}

\pagestyle{myheadings}

\markright{}

\selectlanguage{english}%
\renewcommand{\thepage}{\arabic{page}}

\setcounter{page}{1}

\setlength{\parskip}{7.5pt}

\selectlanguage{spanish}%

\chapter*{{\large Introducci�n}}

\selectlanguage{english}%
\addcontentsline{toc}{chapter}{Introducci�n}\foreignlanguage{spanish}{\thispagestyle{empty} }

\selectlanguage{spanish}%
Venetecnolog�a, empresa integrante del GRUPO VENEMERGENCIA, se encarga
del desarrollo de soluciones tecnol�gicas en el �rea de la salud,
para atender las necesidades de las empresas constituyentes del GRUPO
VENEMERGENCIA. En este sentido, Veneocupacional, empresa de atenci�n
m�dica ocupacional, requiere una aplicaci�n que apoye el proceso de
gesti�n de consultas m�dicas, para mejorar los tiempos de atenci�n
efectiva a los pacientes, la generaci�n y entrega de informes m�dicos,
y la generaci�n de reportes estad�sticos sobre patolog�as recurrentes.
Estas tareas se realizan de manera poco eficiente, y requieren de
procesos que automaticen y optimicen los tiempos de ejecuci�n de ellas,
debido al incremento de la cantidad de pacientes evaluados. �sto motiv�
a Venetecnolog�a a ejecutar un proyecto de pasant�a larga que atendiera
estas necesidades.

A continuaci�n, se describen los antecedentes, el planteamiento y
soluci�n del problema, que justifican el proyecto de pasant�as desarrollado.
Adem�s, se exponen los objetivos generales y espec�ficos que se busca
alcanzar en este desarrollo, y el alcance que abarca este proyecto.
Esta informaci�n tiene la intenci�n de que el lector comprenda el
contexto general del proyecto y conozca el prop�sito del mismo.

\vspace{7.5pt}


\noindent \textbf{Antecedentes}

\vspace{7.5pt}


El GRUPO VENEMERGENCIA se encuentra constituido por un conjunto de
empresas cuya misi�n es crear cultura de prevenci�n en el �rea de
salud y asistencia m�dica a trav�s del adiestramiento de comunidades
en primeros auxilios, distribuci�n de botiquines de primeros auxilios,
servicio de atenci�n m�dica, param�dica, gesti�n de salud ocupacional
para empresas y desarrollo de aplicaciones tecnol�gicas que mejoren
los procedimientos m�dicos. El grupo se encuentra constituido por
las empresas Venemergencia AG, C.A., Venelab C.A., Veneocupacional
C.A., Venetecnolog�a C.A. y la Fundaci�n Venemergencia, cada una de
las cuales se encarga de cubrir un �rea espec�fica dentro del esquema
de negocio.

Venetecnolog�a surge a principios del a�o 2010 ``como respuesta al
d�ficit de soluciones tecnol�gicas para el sector salud, espec�ficamente
software especializado y personalizado. El objetivo principal es integrar
las distintas l�neas de negocio en un solo sistema, por motivos de
eficiencia y mejor manejo del negocio. Otros objetivos incluyen el
desarrollo de soluciones para fomentar la prevenci�n dentro del campo
salud''~\cite{Venemergencia}.

Para mediados del mismo a�o, Venetecnolog�a decide iniciar el desarrollo
de dos proyectos que permitir�an modernizar y automatizar las operaciones
relacionadas con los traslados de ambulancias as� como tambi�n la
solicitud de citas y ex�menes de laboratorio tanto para clientes particulares
como para corporativos (empresas afliliadas a los servicios de salud
ocupacional). Es as� como se inicia el desarrollo del Sistema de Ambulancias,
el cual ser�a utilizado por el personal param�dico de Venemergencia
para encargarse de recibir y gestionar todas las solicitudes de traslados
desde Caracas a cualquier lugar en el pa�s, y el Sistema de Citas
de Laboratorio para ser utilizado dentro de Venelab para gestionar
sus actividades y que a su vez funcionar�a como un portal web que
le permite a los pacientes solicitar citas y ex�menes de laboratorio,
desde la comodidad de sus hogares, al igual que poder revisar los
resultados obtenidos a trav�s del mismo.

En este mismo a�o, tambi�n se crea Veneocupacional con el fin de separar
la l�nea de negocios de medicina ocupacional y atenci�n primaria en
salud para maximizar los beneficios que se ve�an en esa oportunidad
de mercado. Venetecnolog�a decide entonces iniciar el desarrollo del
Sistema Veneocupacional para dar apoyo en esta �rea.

\vspace{7.5pt}


\noindent \textbf{Planteamiento del problema}

\vspace{7.5pt}


Debido a la r�pida expansi�n que ha tenido la empresa dentro del mercado,
lo que trae como consecuencia un incremento en la demanda de los servicios
ofrecidos, se hace necesario automatizar y mejorar la eficiencia de
los procesos relacionados en el manejo de consultas m�dicas. El aumento
en la afluencia de pacientes en el consultorio de Veneocupacional,
ha llevado a necesitar herramientas tecnol�gicas para evitar tiempos
de respuesta inadecuados y falta de atenci�n a los pacientes, para
as� continuar proporcionando un servicio de calidad a sus usuarios. 

Es as� como Venetecnolog�a, se propone desarrollar un sistema de atenci�n
m�dica. Con dicha aplicaci�n, se busca no s�lo facilitar a un m�dico
el acceso a informaci�n de los pacientes a los que trata, sino tambi�n,
mejorar su eficiencia en el desarrollo de las consultas m�dicas, y
a su vez, liberarlo de tareas que pueden automatizarse con la informaci�n
recolectada en la misma, como la generaci�n de informes m�dicos. 

\vspace{7.5pt}


\noindent \textbf{Soluci�n del problema}

\vspace{7.5pt}


Se requiere desarrollar un sistema de informaci�n que permita a los
m�dicos gestionar las consultas m�dicas de sus pacientes. El sistema
debe ser capaz de llevar el registro de los pacientes, antecedentes
m�dicos e historias.

El sistema debe incluir un m�dulo de inicio de sesi�n de usuarios,
que defina la vista de informaci�n por jerarqu�as. El sistema debe
mostrar informaci�n dependiendo del tipo de usuario conectado y la
permisolog�a para el mismo. 

Debe contar con una base de datos de medicamentos, los cuales pueden
ser prescritos por el m�dico a sus pacientes. Se debe poder imprimir
reportes de las consultas. Se debe poder solicitar ex�menes y perfiles
de laboratorio en el transcurso de una consulta. 

La aplicaci�n debe permitir ingresar la informaci�n de manera f�cil,
intuitiva y r�pida. 

Sobre la propuesta planteada, se realiz� el dise�o e implementaci�n
de una aplicaci�n que integrar� los requerimientos de la empresa:
gesti�n de consultas de pacientes, solicitud de ex�menes de laboratorio
y exposici�n de reportes estad�sticos de patolog�as recurrentes.

Se propone iniciar el desarrollo utilizando el marco de desarrollo
(\textsl{framework}) de PHP, denominado CakePHP, en el entorno de
desarrollo integrado (IDE) Aptana para dise�ar e implementar los elementos
pertenecientes al cliente y para la comunicaci�n con los servidores.
Se utilizar� el lenguaje de programaci�n Javascript, mediante el \textsl{framework}
jQuery, para el desarrollo de las comunicaciones basadas en AJAX,
y para la generaci�n de contenidos din�micos en el lado del cliente.

Para la generaci�n de informes m�dicos y reporte de interconsultas,
se plantea utilizar la librer�a TCPDF, desarrollada para trabajar
con PHP, que provee de una interfaz para la generaci�n de documentos
en formato PDF y para el despliege en Flash de los reportes estad�sticos,
se emplear� la librer�a \textsl{Open Flash Chart 2}, desarrollada
en PHP, para generar gr�ficos y cuadros estad�sticos.

El manejo de informaci�n en cuanto a consultas y registros se realizar�a
sobre los servidores con los que trabajan actualmente en la empresa
y de ser necesario se aplicar�an ajustes para lograr que la nueva
aplicaci�n pueda integrarse sin problemas con los dem�s sistemas.
Estos servidores poseen bases de datos que funcionan bajo el manejador
relacional MySQL.

A continuaci�n, se presentan los objetivos del proyecto de pasant�a
larga.

\vspace{7.5pt}


\noindent \textbf{Objetivo general}\vspace{7.5pt}


Desarrollar un sistema que permita a los m�dicos atender las consultas
de sus pacientes, paso a paso y de manera intuitiva, registrando la
informaci�n recopilada en la consulta, y permitiendo que sea visualizada
posteriormente.\vspace{7.5pt}


\noindent \textbf{Objetivos espec�ficos}\vspace{7.5pt}

\begin{itemize}
\item Establecer la visi�n del sistema.
\item Especificar los requerimientos del sistema.
\item Implementar el login para el manejo de consultas.
\item Dise�ar la arquitectura del sistema.
\item Implementar m�dulos para antecedentes m�dicos, resultados de ex�menes
f�sicos, funcionales y paracl�nicos (ex�menes de laboratorio y estudios
de gabinete) y prescripci�n de medicamentos, e interconsultas.
\item Integrar el sistema de atenci�n con los sistemas ya existentes de
la empresa Venemergencia (sistema de gesti�n de citas de laboratorio).
\item Implementar los reportes de patolog�as de los pacientes, posibilidad
de imprimir reportes de las consultas.
\item Documentar el sistema.
\end{itemize}
\vspace{7.5pt}


\noindent \textbf{Alcance}

\vspace{7.5pt}


Desarrollar una aplicaci�n de car�cter m�dico para gesti�n de consultas
m�dicas. Esta aplicaci�n debe manejar la siguiente informaci�n relacionada
con los pacientes: antecedentes m�dicos, h�bitos, consultas previas,
resultados de ex�menes paracl�nicos, registro de ex�menes f�sicos
y funcionales, diagn�stico de patolog�as, prescripci�n de medicamentos,
entre otras.

Un aspecto importante, es la posibilidad de solicitar ex�menes y perfiles
m�dicos al laboratorio Venelab, perteneciente al GRUPO VENEMERGENCIA.
La aplicaci�n debe estar desarrollada para ser utilizada a trav�s
de un navegador web, para simplificar su utilizaci�n y asi no requerir
ning�n procedimiento de instalaci�n que dificulte esta tarea; requiere
estar disponible en todo momento para cualquier m�dico que cuente
con un registro de datos de paciente con la empresa, siempre y cuando
se disponga de una conexi�n a internet. El trasfondo de desarrollar
una aplicaci�n como tal es buscar de alguna manera crear una base
de datos de antecedentes m�dicos de pacientes a la cual un med�co
pueda acceder en el momento que quiera sin tener que perder tiempo
esperando a que se ubique el historial m�dico archivado; tiempo que
puede ser invertido en ir tratando al paciente, la idea es agilizar
y dar apoyo al proceso de asistencia para mejorar la calidad del servicio.

La aplicaci�n debe estar dise�ada para manejar cualquier tipo de informaci�n
m�dica relevante. Debe contar con una interfaz intuitiva y amigable
adapt�ndose a las exigencias de los m�dicos quienes deben poder ingresar
la informaci�n de manera f�cil y r�pida, entorpeciendo lo menos posible
su labor.

Es relevante indicar sobre el objetivo relacionado a la integraci�n
con el sistema de gesti�n de citas de laboratorio, que su alcance
fue modificado debido a la reestructuraci�n de dicho sistema, que
ejecuta la empresa Venectecnolog�a, donde se plantea utilizar una
nueva aplicaci�n para realizar estas actividades. En consecuencia,
la integraci�n se simplific� a la realizaci�n de un proceso de solicitud
de citas via correo electr�nico, a trav�s de los datos capturados
en las consultas m�dicas, adem�s, de el desarrollo de un m�dulo de
carga de archivos e im�genes, que permite asociar documentos digitalizados
a una consulta m�dica.

El alcance de la aplicaci�n a desarrollar, consiste en la entrega
de un prototipo 80\% funcional, a nivel de casos de uso. La aplicaci�n
debe cubrir los aspectos operativos necesarios que le permitan a un
m�dico la utilizaci�n de la herramienta de manera confiable y segura
proporcion�ndole la informaci�n necesaria para atender a un paciente
de manera inmediata, teniendo en cuenta la preservaci�n y confidencialidad
de los datos a los que se tendr� acesso.

\selectlanguage{english}%
\pagebreak{}\selectlanguage{spanish}%




\chapter{Entorno Empresarial}

\thispagestyle{empty} 

Este cap�tulo tiene como objetivo dar a conocer el entorno empresarial
en el cual se desarrolla el proyecto de pasant�as. Busca mostrar la
labor del GRUPO VENEMERGENCIA, y las empresas que lo componen, dando
una idea de sus objetivos, misi�n y aspectos hist�ricos relevantes. 


\section{Descripci�n general}

El Grupo Venemergencia se consolida como un holding en el a�o 2010
para integrar las actividades de las empresas Venemergencia AG, C.A.,
Venelab C.A., Veneocupacional C.A., Venetecnolog�a C.A. y la Fundaci�n
Venemergencia. ``El desarrollo y crecimiento sostenible es el objetivo
del Grupo Venemergencia para el futuro pr�ximo. Sus planes responden
a la visi�n clara de continuar transformando el mercado de los servicios
de salud en Venezuela''~\cite{CronologiaV}. 


\subsection{Venetecnolog�a}

A continuaci�n se presentar� una breve descripci�n de Venetecnolog�a
C.A., empresa parte del Grupo Venemergencia, encargada del desarrollo
de aplicaciones para satisfacer los requerimientos y necesidades existentes
en el �rea de la salud. 


\subsubsection{Historia}

En el a�o 2010, ``se crea Venetecnolog�a como respuesta al d�ficit
de soluciones tecnol�gicas para el sector salud, especificamente software
especializado y personalizado''~\cite{CronologiaV}. Venetecnolog�a
se plantea un �nico objetivo de abarcar el �rea de soluciones tecnol�gicas
para el sector salud.


\subsubsection{Misi�n}

Su objetivo principal es integrar las distintas l�neas de negocio,
de tal forma que se puedan crear aplicaciones o sistemas que le permitan
al personal manejar de manera eficiente las operaciones diarias, as�
como tambi�n ofrecer soluciones tecnol�gicas en los diferentes centros
especializados en atenci�n m�dica.


\subsubsection{Visi�n}

Adentrarse en el �rea de la telemedicina para crear las mejores soluciones
tecnol�gicas adaptadas a las necesidades de los pacientes, permitiendo
atacar los problemas existentes en la actualidad en los diferentes
centros de atenci�n m�dica del pa�s.


\section{Estructura organizacional}

Venetecnolog�a es una empresa peque�a. Sus operaciones dependen de
la Gerencia de Tecnolog�a, quien est� a cargo de la supervisi�n de
los proyectos en desarrollo, y de la Direcci�n General del Grupo Venemergencia.
Un peque�o grupo de desarrolladores, se encarga del desarrollo de
las aplicaciones y proyectos de la empresa, entre los que se encuentra
el proyecto de pasant�as a ejecutar. 

La aplicaci�n se desarroll� en el �rea de Venetecnolog�a, sin embargo,
la misma dar� apoyo al sistema de citas de laboratorio de Venelab,
se utilizar� para facilitar el proceso de consultas m�dicas en Veneocupacional
y trabajar� en conjunto con el sistema de telemedicina y asistencia
m�dica remota, v�a celular ContactaMed; siendo as� una aplicaci�n
que sirve de puente entre estas ramas del negocio.

En la Figura 1.1, se muestra la estructura organizacional de Venetecnolog�a.

\vspace*{1cm}


\begin{center}
\begin{figure}[h]
\centering{}\includegraphics[scale=0.6]{EstructuraVenetec}\caption{Estructura organizacional Venetecnolog�a C.A.}
\end{figure}

\par\end{center}

\newpage{}


\include{capitulo2}


\chapter{Marco Te�rico y Tecnol�gico}

\thispagestyle{empty} 

A continuaci�n, se presentan los conceptos te�ricos relevantes sobre
los cuales se fundament� el proyecto de pasant�a. Adicionalmente,
se definir�n aquellos conceptos, componentes y nociones que est�n
asociados con los diferentes servicios que brinda la empresa y que
fueron beneficiados con el proyecto, tambi�n se dar�n a conocer qu�
herramientas inform�ticas fueron utilizadas para llevar a cabo el
desarrollo del mismo. 


\section{Conceptos b�sicos}

En esta secci�n, se definen los conceptos te�ricos importantes y aspectos
del negocio que forman parte del desarrollo del proyecto. Se pretende
introducir al lector en los t�rminos manejados en el �rea m�dica,
que son relevantes para la gesti�n de consultas m�dicas. Estos conceptos
b�sicos se obtuvieron mediante entrevistas directamente con los directores
del GRUPO VENEMERGENCIA, m�dicos y conocedores del negocio.


\subsection{Consultas m�dicas}

Una consulta m�dica es un procedimiento en el cual un paciente invierte
tiempo junto a un profesional de la medicina en un lugar o espacio
determinado (un consultorio o el domicilio de quien presenta problemas
de salud); en tal procedimiento, el m�dico brinda sus puntos de vista
acerca del estado de salud del paciente y suministra recomendaciones
a seguir para mantenerlo o mejorarlo.

Las consultas m�dicas realizadas por un m�dico generalmente quedan
registradas en un documento de car�cter informativo, cient�fico y
legal conocido como historia cl�nica o m�dica del paciente. En la
historia cl�nica queda constancia del estado de salud del paciente,
as� como tambi�n de la actuaci�n que ha tenido un m�dico con el mismo.

Un m�dico, durante una consulta m�dica, recopila informaci�n sobre
la situaci�n del paciente, ya sea a trav�s de preguntas de un cuestionario
m�dico, como a trav�s de la realizaci�n de diferentes tipos de ex�menes.
Esta informaci�n es organizada y analizada para poder emitir un diagn�stico
al cual, por lo general, se le dise�ar� un plan de acci�n para hacer
frente a las irregularidades presentes, ya sea por medio de tratamiento
a partir de prescripci�n de medicinas o asesoramiento con h�bitos
f�sicos, alimenticios, etc.

Dentro de la empresa se realizan consultas m�dicas a trav�s de la
l�nea de negocios de Veneocupacional, la cual se encarga de brindar
servicio de asistencia en salud ocupacional. En la actualidad se realizan:
\begin{itemize}
\item Consultas anuales.
\item Consultas de pre-empleo.
\item Consultas de egreso.
\item Consultas de emergencia.
\item Consultas para conferimiento de reposo.
\item Consultas por accidentes laborales.
\item Consultas por reubicaci�n laboral.
\item Consultas prevacacionales.
\item Consultas postvacacionales.
\item Consultas por operativos de salud.
\item Consultas para obtenci�n de certificado m�dico vial.
\end{itemize}
~

Con el desarrollo del proyecto de pasant�a, se desea interactuar con
consultas desarrolladas a trav�s del uso de la aplicaci�n para dispositivos
m�viles \textsl{ContactaMed,} por lo que se a�adir�an a la lista anterior:
\begin{itemize}
\item Consultas realizadas a trav�s de una llamada telef�nica.
\item Consultas realizadas a trav�s de un mensaje de texto.
\item Consultas realizadas a trav�s de un correo electr�nico.
\end{itemize}

\subsection{Interconsultas}

Una interconsulta es un documento remitido por un m�dico en el cual
se indica a un paciente que requiere evaluaci�n de un determinado
especialista en el campo de la salud.


\subsection{Historia cl�nica}

La historia cl�nica de un paciente es un documento en el cual se lleva
un registro del estado de salud del mismo respaldado por las diferentes
pr�cticas que, a lo largo del tiempo, un m�dico ha realizado. La funci�n
primordial de una historia cl�nica es poder llevar un seguimiento
de la situaci�n de un paciente y permitirle a un m�dico realizar un
diagn�stico a la hora de presentarse alguna patolog�a, sea recurrente
o no.

Una historia cl�nica se encuentra compuesta por los antecedentes m�dicos,
h�bitos, consultas y ex�menes que se le han realizado previamente
a un paciente.


\subsection{Antecedentes m�dicos}

Consiste en la recopilaci�n de informaci�n acerca de la salud de una
persona. Los antecedentes m�dicos contienen informaci�n relevante
sobre alerg�as a medicamentos, ambientales, procedimientos quir�rgicos
previos de importancia, problemas cardiol�gicos, hipertensi�n, entre
otros.

Los antecedentes m�dicos pueden categorizarse en personales y familiares.
Los personales se refieren a las condiciones previamente mencionadas
presentes en el paciente a evaluar, y los familiares, a condiciones
presentes en parientes cercanos al sujeto a evaluar, que indiquen
una propensi�n a dichas enfermedades.


\subsection{H�bitos}

Se refiere a las actividades, costumbres, y actitudes habituales del
paciente, que son relevantes para determinar su estilo de vida, y
que pueden indicar propensi�n a alguna patolog�a o condici�n m�dica
determinada. Por ejemplo, si el paciente fuma, consume alcohol, realiza
ejercicio peri�dicamente, entre otras.


\subsection{Ex�menes}

Constituyen las pruebas y los estudios realizados por m�dicos y otros
profesionales de la salud, para determinar si un paciente presenta
una enfermedad, evaluar si presenta riesgos de problemas m�dicos en
el futuro, o servir de seguimiento del progreso de una afecci�n conocida
que ya presente el mismo. 

Los ex�menes que se realizan o eval�an durante una consulta m�dica
son: examen funcional, examen f�sico y ex�menes paracl�nicos.


\subsubsection{Examen funcional}

Constituye el interrogatorio de un grupo de s�ntomas (fen�menos cl�nicos
subjetivos) � funciones de �rganos y sistemas del paciente con sus
caracter�sticas. Explora las funciones de �rganos y sistemas, a fin
de determinar la tendencia o manifestaci�n de las funciones de �stos,
que puedan indicar problemas existentes en el paciente o alg�n tipo
de enfermedad, asociado o no al motivo original de la consulta.


\subsubsection{Examen f�sico}

Es el procedimiento de estudio del cuerpo de un paciente para determinar
la presencia o ausencia de problemas f�sicos mediante el reconocimiento
de alteraciones o signos producidos por una enfermedad, vali�ndose
de los sentidos y los instrumentos dise�ados para tal fin. En el examen
f�sico intervienen cuatro (4) m�todos de exploraci�n cl�nica: inspecci�n,
palpaci�n, auscultaci�n y percusi�n.


\subsubsection{Examen paracl�nico}

Se refiere a todos aquellos estudios o ex�menes que son requeridos
por un m�dico a un paciente para comprobar o descartar un diagn�stico.
El m�dico recurre a laboratorios o m�dicos especializados para realizar
los estudios que requiere para hacer una evaluaci�n del paciente.
Entre estos ex�menes podemos encontrar: ex�menes de laboratorio, radiol�gicos,
tomograf�as, entre otros.


\subsection{Diagn�sticos}

Un diagn�stico se refiere a la ``calificaci�n que da el m�dico a
la enfermedad seg�n los signos que advierte''~\cite{RaeDiagnostico}.
En base a los s�ntomas y signos mostrados por el paciente en los los
ex�menes que le son efectuados, el m�dico determina si el mismo presenta
una patolog�a espec�fica.


\subsection{Proceso de Atenci�n de Pacientes}

El proceso de atenci�n de pacientes requiere de la ejecuci�n de cada
uno de los conceptos te�ricos previos. Se presenta como se lleva a
cabo en la actualidad y en qu� actividades existen problemas y hacia
cu�les de ellos apunta la soluci�n a desarrollar.

En la figura 2.1, se presenta un diagrama de actividades, que muestra
el desarrollo del proceso de atenci�n de pacientes.

\begin{center}
\begin{figure}[h]
\begin{centering}
\includegraphics[scale=0.7]{ActividadesAtenci�n}
\par\end{centering}

\caption{Diagrama de actividades del proceso de atenci�n de pacientes.}
\end{figure}

\par\end{center}

Los problemas del proceso de atenci�n de pacientes ocurren principalmente,
en la Recopilaci�n de Antecedentes y H�bitos del paciente, donde se
utiliza tiempo para capturar los antedecendentes, personales y familiares
de los pacientes en cada una de las consultas; tambi�n ocurren en
la solicitud de interconsultas y citas de laboratorio, y en la generaci�n
de informes m�dicos, que deben ser realizados de manera manual, por
lo que se debe invertir tiempo de los m�dicos en su redacci�n, pudiendo
ser tomados directamente de los datos capturados en la consulta m�dica. 


\section{Herramientas inform�ticas}

En esta secci�n, se introduce al lector en los conceptos tecnol�gicos
inform�ticos que forman parte del desarrollo del proyecto de pasant�as.
Se pretende reflejar no s�lo sus definiciones, sino tambi�n, el papel
que dichas herramientas juegan en la arquitectura del sistema.


\subsection{Modelo Cliente - Servidor}

La arquitectura cliente-servidor permite al usuario en una m�quina,
llamada el cliente, requerir alg�n tipo de servicio de una m�quina
a la que est� unido, llamado el servidor, mediante una red como una
LAN (Red de Area Local) o una WAN (Red de Area Mundial). Estos servicios
pueden ser peticiones de datos de una base de datos, de informaci�n
contenida en archivos o los archivos en s� mismos, o peticiones de
imprimir datos en una impresora asociada~\cite{ClienteServidor}.

El modelo cliente-servidor se fundamenta en la idea de que un proveedor
de datos por si mismo puede abastecer las solicitudes de numerosos
clientes que requieren de sus servicios.

La aplicaci�n a desarrollar es un ejemplo de un sistema basado en
este modelo. Cada cliente estar� representado por un navegador web,
que realizar� solicitudes al servidor mediante las vistas desplegadas
por el sitio web; y el servidor, que contendr� la aplicaci�n de gesti�n
de consultas m�dicas, y atender� cada una de las solicitudes del cliente.


\subsection{Modelo Vista Controlador (MVC)}

El patr�n de dise�o MVC define una estructura de desarrollo de software
que separa las diferentes tareas que deben ser atendidas en una aplicaci�n,
con la finalidad de mejorar la mantenibilidad del c�digo. 

MVC fue concebido en 1978 como la soluci�n de dise�o para un problema
particular. Su objetivo de primordial fue apoyar el modelo mental
del usuario del espacio de la informaci�n relevante y que el usuario
pueda inspeccionar y editar esta informaci�n~\cite{OsloMVC}.

El objetivo de usar este patr�n de dise�o, es separar separa los datos
de una aplicaci�n, la interfaz de usuario, y la l�gica de control
en tres componentes distintos. El patr�n MVC se ve frecuentemente
en aplicaciones Web, donde la vista es la p�gina HTML y el c�digo
que provee de datos din�micos a la p�gina, el modelo es el Sistema
de Gesti�n de Base de Datos y el controlador representa la L�gica
de negocio~\cite{LibroJoomla}.

En el proyecto, se emple� el MVC mediante el sistema de vistas, controladores
y modelos que provee CakePHP, donde, efectivamente, se diferencian
las tareas de acceso a los datos, la l�gica de programa y la interfaz
gr�fica; tal caracter�stica facilita la labor del desarrollador, ya
que le permite mantener un control de los problemas que pueden estar
ocurriendo en las fases de desarrollo, al conocer en cual componente
exactamente est� sucediendo, adem�s facilita las tareas donde dependiendo
de los datos, se debe acceder a una u otra vista determinada, permitiendo
a los modelos tener diferentes vistas, que pueden ser accedidas en
los diferentes casos.

En la figura 2.2, se observa un diagrama de la interacci�n entre las
distintas partes de una aplicaci�n basada en el patr�n MVC, mostrando
la comunicaci�n entre el Modelo (\textsl{Model}) y Controlador (\textsl{Controller}),
el Controlador comunic�ndose con la Vista (\textsl{View}) y la interacci�n
del Cliente (\textsl{Client}) con la aplicaci�n para acceder a las
operaciones presentes en el Controlador.

\begin{center}
\includegraphics[scale=0.7]{basic_mvc}
\begin{figure}[h]
\caption{Arquitectura b�sica del patr�n MVC~\cite{MVCCakePHP}}


\end{figure}

\par\end{center}


\subsection{PHP: Hypertext Preprocessor (PHP)}

PHP, acr�nimo de \textquotedbl{}PHP: Hypertext Preprocessor\textquotedbl{},
es un lenguaje \textquotedbl{}Open Source\textquotedbl{} interpretado
de alto nivel, especialmente pensado para desarrollos web y el cual
puede ser incrustado en p�ginas HTML. La mayor�a de su sintaxis es
similar a C, Java y Perl y es f�cil de aprender. La meta de este lenguaje
es permitir escribir a los creadores de p�ginas web, p�ginas din�micas
de una manera r�pida y f�cil, aunque se pueda hacer mucho m�s con
PHP~\cite{ManualPHP}.

En el proyecto se emplea, fundamentalmente, este lenguaje para el
desarrollo de la aplicaci�n, apoy�ndose en el Marco de Desarrollo
(\textsl{framework}) CakePHP.


\subsection{CakePHP}

CakePHP es un marco de desarrollo (\textsl{framework}) r�pido para
PHP, libre, de c�digo abierto. Se trata de una estructura que sirve
de base a los programadores para que �stos puedan crear aplicaciones
Web~\cite{QueEsCakePHP}. 

El objetivo principal de CakePHP es dar al programador herramientas
para trabajar de manera estructurada y r�pida, sin p�rdida de flexibilidad.
Es una herramienta muy �til, que provee al desarrollador de caracter�sticas
interesantes como:
\begin{itemize}
\item Ayudantes para AJAX, Javascript, formularios.
\item Componentes de Seguridad, Sesi�n, Email.
\item Listas de Control de Acceso.
\item Configuraci�n simplificada.
\end{itemize}
~

CakePHP incluye las clases Controlador, Modelo y Vista, pero tambi�n
incluye otras clases y objetos que hacen que el desarrollo en MVC
sea un poco m�s r�pido y agradable. Los Componentes (\textsl{Components}),
Comportamientos (\textsl{Behaviors}), y Ayudantes (\textsl{Helpers})
son clases que proporcionan extensibilidad y reusabilidad; agregan
r�pidamente funcionalidad a las clases base MVC de las aplicaciones~\cite{EstructuraCakePHP}.

En la figura 2.3, se muestra un diagrama de la interacci�n de las
diferentes clases y objetos utilizados por CakePHP para implementar
su versi�n del MVC.

\begin{center}
\includegraphics[scale=0.6]{typical-cake-request}
\begin{figure}[h]
\caption{Petici�n t�pica de una aplicaci�n en CakePHP~\cite{CakePHP-Request}}
\end{figure}

\par\end{center}

CakePHP es el entorno de desarrollo principal de la aplicaci�n a desarrollar,
englobando las tareas de conexi�n con la base de datos, procesamiento
de informaci�n y presentaci�n al usuario, mediante su esquema de Modelos,
Vistas y Controladores. �sto permite simplificar en gran medida la
configuraci�n de la aplicaci�n y el envio de informaci�n entre cada
una de las capas del MVC, adem�s de facilitar gran cantidad de tareas
al desarrollador mediante las diferentes funciones de alto nivel y
\textsl{Helpers} que provee este \textsl{framework}. 


\subsection{Asynchronous JavaScript and XML (AJAX)}

JavaScript as�ncrono y XML (Asynchronous JavaScript and XML) o AJAX,
se refiere a un conjunto de t�cnicas utilizadas en el desarrollo web,
basadas en el modelo cliente-servidor, empleadas para la construcci�n
de aplicaciones interactivas, que se ejecutan a nivel del cliente,
interactuando de manera as�ncrona con el servidor, para la solicitud
de datos. De esta forma, es posible generar cambios en las vistas
presentadas al usuario sin necesidad de recargar la p�gina, con lo
que se mejora la interactividad y usabilidad de las aplicaciones.

Las aplicaciones construidas con AJAX eliminan la recarga constante
de p�ginas mediante la creaci�n de un elemento intermedio entre el
usuario y el servidor. La nueva capa intermedia de AJAX mejora la
respuesta de la aplicaci�n, ya que el usuario nunca se encuentra con
una ventana del navegador vac�a esperando la respuesta del servidor~\cite{IntroAjax}.

En la Figura 2.3, se presenta una comparaci�n entre el procesamiento
de una aplicaci�n web s�ncrona cl�sica y una aplicaci�n desarrolla
mediante las t�cnicas as�ncronas de AJAX.

\begin{center}
\includegraphics[scale=0.5]{ajax-fig2_small_es}
\begin{figure}[h]
\caption{Comparaci�n de procesamiento entre una aplicaci�n web cl�sica y una
aplicaci�n web basada en AJAX}
~\cite{Ajax}
\end{figure}

\par\end{center}

~

La utilizaci�n de AJAX es fundamental para el desarrollo del proyecto,
sobre todo en el proceso de creaci�n y gesti�n de consultas m�dicas,
que, b�sicamente, requiere procesar gran cantidad de formularios diferentes,
sin recargar la vista de la consulta m�dica, para permitir al usuario
ejecutar tareas, y adem�s evitando el reenv�o por parte del servidor
de informaci�n que ya se encuentra presente en la p�gina, aligerando
as�, la carga de operaciones sobre el mismo.



\chapter{Marco Metodol�gico}

\thispagestyle{empty} 

El desarrollo de este proyecto se lleva a cabo mediante la aplicaci�n
de la metodolog�a Proceso Unificado �gil (\textit{Agile Unified Process},
AUP o AgileUP). Esta metodolog�a fue desarrollada por Scott W. Ambler.

AUP es una versi�n simplificada del Proceso Unificado de Rational
\textit{(Rational Unified Process} o RUP), que presenta una aproximaci�n
sencilla con t�cnicas orientadas al desarrollo de software �gil, pero
manteniendo muchos conceptos importantes de RUP~\cite{AUP}.

AUP emplea t�cnicas �giles, incluyendo Desarrollo Guiado por Pruebas
(\textit{Test Driven Development} o TDD), modelado �gil, gesti�n �gil
de cambios y reconstrucci�n de bases de datos. Algunas de estas t�cnicas
formaron parte integral del desarrollo del proyecto.

Para la documentaci�n y registro de los requerimientos del sistema,
se utilizar�n las plantillas de documentos para Documento de Visi�n
del Sistema, Especificaci�n de Requerimientos de Software (ERS) y
Documento de Arquitectura de Software (DAS), pertenecientes a la metodolog�a
RUP.

Se emplea AUP, debido a que, al ser una metodolog�a �gil, presenta
un equilibrio entre las etapas de dise�o, planificaci�n y documentaci�n
del sistema, con la implementaci�n del mismo, por lo tanto, se sigue
manteniendo un desarrollo estructurado y coherente, pero con una mayor
flexibilidad para el programador, tanto para en tiempo efectivo de
implementaci�n, como en el manejo de cambios en el sistema a trav�s
del proceso de desarrollo.


\section{Fases de AUP}

La metodolog�a de AUP, de igual manera que RUP, propone dividir el
desarrollo de software en cuatro (4) fases, cada una de las cuales
comprende t�cnicas y disciplinas de desarrollo, que permiten iterar
sobre ellas hasta lograr el cumplimiento de los objetivos planteados.
Al ser una metodolog�a flexible y �gil, AUP permite que las t�cnicas
que componen cada fase, se puedan adecuar a las necesidades y realidades
del proyecto. 

A continuaci�n, se muestra un gr�fico con las fases de la metodolog�a
AUP (Figura 3.1).

\begin{center}
\begin{figure}[h]
\centering{}\includegraphics[scale=0.65]{AUPfases}\caption{Fases y disciplinas de la metodolog�a AUP \cite{AUP}.}
\end{figure}

\par\end{center}


\subsection{Inicio (\textit{Inception})}

El objetivo es identificar el alcance inicial del proyecto, una arquitectura
potencial de su sistema, el alcance estimado del proyecto, establecer
c�mo ser�n administrados los recursos, y obtener la aceptaci�n del
involucrado.

La administraci�n de los recursos corre por cuenta de Venetecnolog�a
y el Grupo Venemergencia, quienes encargaron el desarrollo de la aplicaci�n,
y adem�s, tambi�n son los usuarios finales de la misma.

De esta manera, las actividades que competen al pasante en esta fase
est�n relacionadas con la definici�n de los requerimientos del sistema
y selecci�n de tecnolog�as a emplear en el desarrollo pr�ximo.

Se establecen claramente unos objetivos para esta fase:
\begin{itemize}
\item Estudio de los sistemas existentes en la empresa, para determinar
si se debe o no interactuar con ellos, y de qu� manera se puede hacer
posible dicha interacci�n.
\item Revisi�n inicial de los requerimientos funcionales del sistema, en
base a la planificaci�n de 20 semanas de ejecuci�n del proyecto.
\item Familiarizarse con los t�rminos y conceptos relacionados al negocio,
que afecten o formen parte integral del proyecto.
\item Definici�n de tecnolog�as y herramientas a utilizar para el desarrollo
del proyecto.
\end{itemize}

\subsection{Elaboraci�n (\textit{Elaboration})}

Esta fase tiene como objetivo principal, definir la arquitectura del
sistema, determinando c�mo estar� estructurada la aplicaci�n, tanto
a nivel de componentes, como a nivel de modelo de datos. Tambi�n contempla
el refinamiento de los requerimientos funcionales, mediante la formalizaci�n
de los mismos, a trav�s de la definici�n de los casos de uso.

Una de las actividades de mayor importacia en esta fase se concentra
en la definici�n del modelo de datos a emplear, que en este proyecto
debe integrar las necesidades nuevas requeridas por los usuarios para
el sistema de gesti�n de consultas m�dicas, como las requeridas en
el desarrollo del sistema de atenci�n de consultas m�dicas, v�a celular,
ContactaMed.

Los objetivos espec�ficos de esta fase son:
\begin{itemize}
\item Formalizaci�n y refinamiento de los requerimientos funcionales del
sistema, mediante la construcci�n del documento de Especificaci�n
de Requerimientos de Software (ERS), y la realizaci�n del diagrama
de casos de uso.
\item Definici�n de los componentes y subcomponentes del sistema.
\item Dise�o y elaboraci�n del Modelo de Datos del sistema, que debe integrar
las necesidades de la aplicaci�n de consultas m�dicas con las del
sistema ContactaMed, que se encuentra en desarrollo.
\item Adiestramiento en las herramientas de programaci�n, lenguajes y \textit{frameworks}
a utilizar en el desarrollo de la aplicaci�n (PHP, CakePHP, jQuery,
Javascript).
\end{itemize}

\subsection{Construcci�n (\textit{Construction})}

El objetivo primordial de esta fase est� constituido por la implementaci�n
de un sistema, de manera incremental, donde, iteraci�n a iteraci�n,
se van desarrollando los componentes y subcomponentes de la aplicaci�n,
bas�ndose en la prioridad establecida de cada uno de los requerimientos
definidos en las fases previas. 

Este desarrollo incremental, se logra a trav�s de la implementaci�n
de m�dulos parciales funcionales, que cumplan con las necesidades
de los requerimientos asociados al desarrollo ejecutado. Tambi�n,
se van aplicando pruebas a trav�s de cada iteraci�n, para comprobar
la estabilidad y confiabilidad del software generado.

Para el desarrollo de la aplicaci�n, se dividi� esta fase en tres
(3) iteraciones para la implementaci�n de los Requerimientos:
\begin{enumerate}
\item \textbf{Primera Iteraci�n: Gesti�n de Usuarios del Sistema.}

\begin{itemize}
\item Dise�o e implementaci�n de m�dulo de registro de pacientes.
\item Dise�o e implementaci�n de m�dulo de registro de usuarios m�dicos
y administrativos.
\item Desarrollo de m�dulo de gesti�n de pacientes y usuarios m�dico-administrativos.
\item Desarrollo de m�dulo de login de usuarios (tanto m�dico-administrativos,
como pacientes).
\item Establecimiento del sistema de Listas de Control de Acceso (\textit{Access
Control Lists}) o ACL, para establecer permisolog�as dependiendo del
tipo de usuario.
\end{itemize}
\item \textbf{Segunda Iteraci�n: Gesti�n de Consultas M�dicas.}

\begin{itemize}
\item Dise�o e implementaci�n de m�dulo de registro de antecedentes y h�bitos.
\item Dise�o e implementaci�n de m�dulo de prescripci�n de medicamentos.
\item Construcci�n de m�dulo de ex�menes funcionales, f�sicos y paracl�nicos.
\item Construcci�n de m�dulo de interconsultas.
\item Dise�o e implementaci�n de m�dulo de solicitud de citas de laboratorio
con Venelab.
\item Dise�o e implementaci�n de m�dulo de carga de archivos e im�genes.
\item Construcci�n de m�dulo de diagn�sticos.
\end{itemize}
\item \textbf{Tercera Iteraci�n: Generaci�n de Reportes y Visualizaci�n
de Consultas. }

\begin{itemize}
\item Dise�o e implementaci�n de m�dulo de Generaci�n de Informes de Consultas.
\item Dise�o e implementaci�n de m�dulo de reporte de patolog�as recurrentes.
\item Construcci�n de m�dulo de visualizaci�n y edici�n de consultas previas.
\end{itemize}
\end{enumerate}

\subsection{Transici�n (\textit{Transition})}

Esta fase tiene como objetivo validar y desplegar el sistema en un
ambiente de producci�n, sin embargo, en el contexto de las veinte
(20) semanas definidas para la ejecuci�n del proyecto, se determin�
que la puesta en producci�n de la aplicaci�n no forma parte del alcance
del mismo. Por lo tanto, en esta pasant�a, el alcance de la fase de
transici�n del sistema se limita a pruebas t�cnicas para determinar
la estabilidad, integridad y buen funcionamiento de la aplicaci�n.

\newpage{}



\chapter*{Conclusiones y Recomendaciones}

\selectlanguage{english}%
\addcontentsline{toc}{chapter}{Conclusiones y Recomendaciones}

\selectlanguage{spanish}%
\vspace{5mm}


Al culminar el proyecto de pasant�as, y evaluar los objetivos generales
y espec�ficos, adem�s de los requerimientos funcionales del sistema,
se aprecia que los mismos se cumplieron satisfactoriamente, y en los
plazos estipulados al inicio del proyecto. Se desarroll� una aplicaci�n
que permite gestionar consultas m�dicas, y generar informes m�dicos
con la informaci�n recopilada directamente de la consulta. Esta aplicaci�n,
apoyar� los procesos y actividades que desarrolla la empresa Veneocupacional
C.A., parte integral del Grupo Venemergencia, encargada de atender
las operaciones de Atenci�n Primaria en Salud Ocupacional (APSO),
y adem�s, a futuro, con ciertas modificaciones, ser colocada al servicio
de diversos centros de atenci�n m�dica.

Se obtuvo un sistema que mantiene la informaci�n de las consultas
de los pacientes, la cual puede ser consultada a futuro, lo que permite
recopilar un amplio historial m�dico, que nutrir� las pr�ximas consultas
del paciente. Tambi�n, permite recabar datos estad�sticos interesantes,
en base a las consultas realizadas, tomando los diagn�sticos realizados,
para as�, detectar la ocurrencia de cierta patolog�a o grupo de patolog�as
en cierta poblaci�n de pacientes, y en un rango temporal definido.
Esto �ltimo es de utilidad, tanto para determinar qu� patolog�as ocurren
con mayor propensi�n en alg�n momento determinado, como para conocer
y estar preparados para obtener suministros relacionados a la atenci�n
de cierta patolog�a.

En cuanto a la generaci�n de informes m�dicos, el sistema elimina
la necesidad de redacci�n manual de los mismos, al extraer los datos
asociados a una consulta m�dica que hayan sido ingresados por el m�dico,
disminuyendo el tiempo necesario para la creaci�n de los informes
m�dicos y su entrega al paciente.

Con respecto a la gesti�n de usuarios, la utilizaci�n del esquema
de Listas de Control de Acceso simplifica en gran medida la configuraci�n
de la aplicaci�n para presentar diferentes contenidos, dependiendo
del tipo de usuario.

La solicitud de citas de laboratorio v�a correo electr�nico, es una
herramienta que permite la comunicaci�n directa entre el paciente
atendido en la consulta m�dica y el personal de Venelab, quien recibir�
la informaci�n de contacto del mismo y los datos de perfiles y ex�menes
que se le solicitan, tomando esta informaci�n directamente de la recopilada
en la consulta m�dica y de los datos del paciente que est�n registrados
en el sistema. Esto permite que, luego, el personal de Venelab se
comunique con los pacientes para confirmar horarios y fechas de las
citas a realizar para los estudios de laboratorio.

Un aspecto clave en el �xito alcanzado en el desarrollo del proyecto
est� ligado a la utilizaci�n de la metodolog�a Agile UP, que permite
ajustarse a las necesidades de un equipo peque�o de desarrolladores,
en donde la utilizaci�n tiempo es un factor vital para el logro de
los objetivos, al ser Agile UP, una metodolog�a �gil, que plantea
un balance entre el dise�o y documentaci�n de los sistemas, y la implementaci�n
de las soluciones, lo que permite dedicar el tiempo apropiado a cada
una de las tareas programadas.

Otra muestra del �xito del proyecto de pasant�as largas, represent�
a nivel personal, el aprendizaje efectivo logrado en su implementaci�n.
Se requiri� adiestramiento en tecnolog�as para el desarrollo de aplicaciones
web, como jQuery, PHP, CakePHP, AJAX que complementan los conocimientos
adquiridos a nivel acad�mico. La experiencia en el patr�n de dise�o
MVC, tambi�n juega un rol fundamental en este aprendizaje, permitiendo
el desarrollo de aplicaciones de un modo m�s coherente y mantenible,
mediante la separaci�n de diferentes tareas (datos, l�gica e interfaz)
en componentes distintos. Todo esto, acompa�ado por el uso de una
metodolog�a de desarrollo como AgileUP, que permite compartir de manera
balanceada el tiempo entre desarrollo y documentaci�n.

Aunque se alcanzaron excelentes resultados en el desarrollo del proyecto
de pasant�as, es importante hacer notar un conjunto de aspectos que
pueden tomarse en cuenta para las pr�ximas etapas en el ciclo de vida
de la aplicaci�n desarrollada, con el fin de mejorar su desempe�o
y capacidades. A continuaci�n, se mencionan estos aspectos:

La aplicaci�n en la actualidad est� orientada a la utilizaci�n via
web, lo que requerir� un servidor para alojar el sistema, y que atienda
las solicitudes de los clientes. Es recomendable que se adquiera un
computador que funcione como servidor dedicado para las aplicaciones.
Actualmente, la empresa maneja sus aplicaciones web a trav�s de servidores
localizados en el exterior, lo que implica la necesidad de conexi�n
a internet para el uso de la aplicaci�n a nivel interno. Un servidor
local dedicado, mejorar�a la confiabilidad y disponibilidad de la
aplicaci�n.

Con vistas de hacer llegar este sistema a otras instituciones del
sector salud en Venezuela, ser�a importante desarrollar a futuro una
aplicaci�n \textsl{standalone}, instalable, que no requiera necesariamente
conexi�n de internet para la gesti�n de consultas m�dicas, en caso
de fallas o problemas de red o conectividad, lo que significar�a tener
un sistema m�s robusto, que pueda atender consultas m�dicas b�sicas
sin conexi�n a una red, y luego, al reestablecerse el servicio de
internet, existan procedimientos de sincronizaci�n con los servidores.

En cuanto a la interfaz gr�fica, visualizaci�n y presentaci�n de la
informaci�n, es posible mejorar estos tres aspectos, que por decisi�n
de negocios y por limitaciones de tiempo fueron dejados de lado en
este proyecto. Se deben agregar herramientas de navegaci�n para facilitar
el acceso a las funciones del sistema y mejorar la presentaci�n de
los formularios.

A nivel de datos, ser�a recomendable estudiar algunas de las tablas
de la base de datos, que almacenan informaci�n en formato JSON. Es
posible mejorar la organizaci�n de los datos en este formato para
hacer m�s manejable y simple la extracci�n de la informaci�n que contienen.

Al seguir estas recomendaciones, se obtendr�a una aplicaci�n m�s estable
y robusta, que proveer�a mecanismos de gesti�n de fallos. Adem�s,
con los cambios a nivel de datos, se mejorar�a la mantenibilidad del
sistema. Con respecto a la interfaz gr�fica, se lograr�a mejorar la
experiencia del usuario, al simplificar el uso de la aplicaci�n y
hacer m�s atractiva a la vista la presentaci�n de la informaci�n del
sistema.

\begin{onehalfspace}
\newpage{}\end{onehalfspace}



\selectlanguage{english}%
\setlength{\parskip}{7pt}

\selectlanguage{spanish}%
\renewcommand{\bibname}{Referencias Bibliogr�ficas}
\begin{thebibliography}{10}
\begin{singlespace}
\bibitem{Venemergencia}Grupo Venemergencia. 2010. VENEMERGENCIA,
�La Excelencia en Primeros Auxilios M�dicos!. Disponible en Internet:
http://venemergencia.com/, consultado el 9 de diciembre de 2011.

\bibitem{CronologiaV}Gonz�lez, A. ``Cronolog�a Grupo Venemergencia''.
Grupo Venemergencia, Documento Interno. (2011).

\bibitem{RaeDiagnostico}Real Academia Espa�ola, ``Diccionario de
la Lengua Espa�ola, Vig�sima Primera Edici�n, Tomo I'', Editorial
Espasa, Madrid, pp. 743. (1992).

\bibitem{ClienteServidor}Elizalde, Manuel. 2001. Modelo Cliente-Servidor
Disponible en Internet: http://www.fismat.umich.mx/\textasciitilde{}anta/tesis/node32.html,
consultado el 3 de enero de 2012.

\bibitem{OsloMVC}Reenskaug, T., ``The Model-View-Controller (MVC).
Its Past and Present'', University of Oslo, Noruega, pp. 1 (2003).
\addcontentsline{toc}{chapter}{Referencias Bibliogr�ficas}

\bibitem{LibroJoomla}Serrats, C., ``Programaci� de components MVC
per Joomla! 1.5.x'', CESI Inform�tica i Comunicacions, Girona, pp.
5 (2008). 
\end{singlespace}

\bibitem{MVCCakePHP}Cake Software Foundation, Inc. 2010. Understanding
Model-View-Controller. Disponible en Internet: http://book.cakephp.org/1.3/en/view/890/Understanding-Model-View-Controller,
consultado el 11 de diciembre de 2011.

\begin{singlespace}
\bibitem{ManualPHP}The PHP Group. 2011. Manual de PHP. Disponible
en Internet: http://www.php.net/manual/es/preface.php, consultado
el 10 de diciembre de 2011.

\bibitem{QueEsCakePHP}Cake Software Foundation, Inc. 2010. �Qu� es
CakePHP y por qu� hay que utilizarlo?. Disponible en Internet: http://book.cakephp.org/1.3/es/view/880/Qu\%C3\%A9-es-CakePHP-y-por-qu\%C3\%A9-hay-que-utilizarlo,
consultado el 10 de diciembre de 2011.

\bibitem{EstructuraCakePHP}Cake Software Foundation, Inc. 2010. Estructura
de CakePHP. Disponible en Internet: http://book.cakephp.org/es/view/893/Estructura-de-CakePHP,
consultado el 12 de diciembre de 2011.
\end{singlespace}

\bibitem{CakePHP-Request}Cake Software Foundation, Inc. 2010. A Typical
CakePHP Request. Disponible en Internet: http://book.cakephp.org/1.3/en/view/898/A-Typical-CakePHP-Request,
consultado el 12 de diciembre de 2011.

\begin{singlespace}
\bibitem{IntroAjax}P�rez, J. E., ``Introducci�n a AJAX'', librosweb.es,
pp. 6 (2008).

\selectlanguage{english}%
\bibitem{Ajax}\foreignlanguage{spanish}{Garret, Jesse. 2005. Ajax:
A New Approach to Web Applications. Disponible en Internet: http://www.adaptivepath.com/ideas/ajax-new-approach-web-applications,
consultado el 15 de diciembre de 2011.}

\selectlanguage{spanish}%
\bibitem{AUP}Ambler, Scott. 2009. The Agile Unified Process (AUP).
Disponible en Internet: http://www.ambysoft.com/unifiedprocess/agileUP.html,
consultado el 21 de diciembre de 2011.
\end{singlespace}

\bibitem{MYSQL}Oracle Corporation. 2011. Why MySQL?. Disponible en
Internet: http://www.mysql.com/why-mysql/, consultado el 2 de enero
de 2012.

\bibitem{PrincipiosCake}Cake Software Foundation, Inc. 2010. Principios
b�sicos de CakePHP. Disponible en Internet: http://book.cakephp.org/1.3/es/view/892/Principios-b\%C3\%A1sicos-de-CakePHP,
consultado el 2 de enero de 2012.

\bibitem{jQuery}The jQuery Project. 2010. jQuery: write less, do
more. Disponible en Internet: http://jquery.com/, consultado el 10
de octubre de 2011.

\bibitem{Cake1.3Api}Cake Software Foundation, Inc. 2010. The API
1.3.X. Disponible en Internet: http://api13.cakephp.org/ , consultado
el 3 de septiembre de 2011.

\bibitem{Cookbook}Cake Software Foundation, Inc. 2010. El manual
. Disponible en Internet: http://book.cakephp.org/1.3/es/view/876/El-manual,
consultado el 3 de septiembre de 2011.

\bibitem{build_Acl}Cake Software Foundation, Inc. 2010. An Automated
tool for creating ACOs. Disponible en Internet: http://book.cakephp.org/1.3/en/view/1549/An-Automated-tool-for-creating-ACOs,
consultado el 5 de septiembre de 2011.\end{thebibliography}



\selectlanguage{english}%
\setlength{\parskip}{7.5pt}

\selectlanguage{spanish}%
\appendix

\chapter{Modelo de Datos del Sistema~\label{chap:Modelo-de-Datos}}

\begin{center}
\includegraphics[scale=0.35]{ER-VeneOcupacional}
\begin{figure}
\caption{Diagrama Entidad - Relaci�n del Sistema.}
\end{figure}

\par\end{center}


\chapter{Diagrama de Casos de Uso}

\begin{center}
\includegraphics[scale=0.5]{CasosUsoSistema2}
\begin{figure}
\caption{Diagrama de Casos de Uso}


\end{figure}

\par\end{center}


\chapter{Documento Visi�n del Sistema~\label{chap:Documento-Visi=0000F3n}}

A continuaci�n, se adjunta el documento Visi�n del Sistema, que presenta
una imagen inicial de las necesidades que ser�n satisfechas con el
desarrollo del proyecto, los interesados, usuarios, detalles del producto,
entre otros. 

\begin{center}
\includepdf[pages={1-},pagecommand={\thispagestyle{myheadings}},scale=0.95]{Documento_Vision_Venetecnologia}
\par\end{center}


\chapter{Especificaci�n de Requerimientos de Software\label{chap:Especificaci=0000F3n-de-Requerimientos}}

A continuaci�n, se adjunta el documento de Especificaci�n de Requerimientos
de Software, que detalla cada uno de los requerimientos y casos de
uso determinados para desarrollar en el proyecto de pasant�as.

\begin{center}
\includepdf[pages=-,pagecommand={\thispagestyle{myheadings}},scale=0.95]{Especificaciones_de_Requerimientos_del_Software}
\par\end{center}


\chapter{Documento de Arquitectura de Software\label{chap:Documento-de-Arquitectura}}

A continuaci�n, se adjunta el Documento de Arquitectura de Software,
que describe la estructura de la aplicaci�n a desarrollar en base
al Modelo de las 4+1 vistas.

\begin{center}
\includepdf[pages={1-},pagecommand={\thispagestyle{myheadings}},scale=0.95]{Documento_Arquitectura_Software}
\par\end{center}

\cleardoublepage{}

\cleardoublepage{}
\end{document}
