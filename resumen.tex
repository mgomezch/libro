\renewcommand{\thepage}{\Roman{page}}

\setcounter{page}{4} 

\begin{center}
\textbf{\large Resumen}
\par\end{center}{\large \par}

\vspace{5mm}


\begin{onehalfspace}
El presente proyecto de pasant�a consiste en el dise�o e implementaci�n
de una aplicaci�n m�dica dise�ada para atender consultas m�dicas a
trav�s de un navegador Web. La empresa VENETECNOLOG�A C.A. (filial
del GRUPO VENEMERGENCIA) tiene como objetivo la creaci�n de esta aplicacion
cuyo fin es permitirle a un m�dico poder atender a sus pacientes,
almacenando la informaci�n de cada consulta m�dica, para as�, mantener
un historial completo del paciente. Este historial permitir�a enriquecer
la informaci�n de los pacientes, lo que puede conllevar al alcance
de diagn�sticos m�s precisos, reinvertir los tiempos de consulta utilizados
para recopilar informaci�n que ya se posee, adem�s mejorar y agilizar
los procedimientos que involucra el desarrollo de una consulta.

La idea es construir un conjunto de formularios, que de manera intuitiva
permitan el desarrollo r�pido y efectivo de una consulta m�dica. Tambi�n,
permitir acceder al m�dico a informaci�n de consultas previas En base
a esta informaci�n el m�dico debe poder registrar consultas, recetar
prescripciones, solicitar citas de laboratorio o un traslado de ambulancia
hasta la ubicaci�n del paciente. La aplicaci�n debe mejorar el uso
del tiempo al momento de atender un paciente.
\end{onehalfspace}

La aplicaci�n tambi�n debe automatizar la generaci�n de reportes e
informes m�dicos, en base a la informaci�n recopilada en cada consulta,
tambi�n, debe presentar una vista de reportes estad�sticos, para obtener
informaci�n acerca de las patolog�as recurrentes diagnosticadas entre
los pacientes que han sido consultados. Esta aplicaci�n debe trabajar
en conjunto con el sistema de telemedicina y gesti�n de consultas
m�dicas v�a celular (ContactaMed), que se encuentra en desarrollo,
y el sistema ya existente de citas de laboratorio (VeneLab). 

\begin{onehalfspace}
La metodolog�a utilizada para la implementaci�n de este proyecto es
Agile UP, una metodolog�a de tipo �gil, con un buen balance entre
la planificaci�n del proyecto y tiempos de desarrollo del software.\newpage{}\end{onehalfspace}

