\renewcommand{\bibname}{Referencias Bibliogr�ficas}
\begin{thebibliography}{10}
\begin{singlespace}
\bibitem{Venemergencia}Grupo Venemergencia. 2010. VENEMERGENCIA,
�La Excelencia en Primeros Auxilios M�dicos!. Disponible en Internet:
http://venemergencia.com/, consultado el 9 de diciembre de 2011.

\bibitem{CronologiaV}Gonz�lez, A. ``Cronolog�a Grupo Venemergencia''.
Grupo Venemergencia, Documento Interno. (2011).

\bibitem{RaeDiagnostico}Real Academia Espa�ola, ``Diccionario de
la Lengua Espa�ola, Vig�sima Primera Edici�n, Tomo I'', Editorial
Espasa, Madrid, pp. 743. (1992).

\bibitem{ClienteServidor}Elizalde, Manuel. 2001. Modelo Cliente-Servidor
Disponible en Internet: http://www.fismat.umich.mx/\textasciitilde{}anta/tesis/node32.html,
consultado el 3 de enero de 2012.

\bibitem{OsloMVC}Reenskaug, T., ``The Model-View-Controller (MVC).
Its Past and Present'', University of Oslo, Noruega, pp. 1 (2003).
\addcontentsline{toc}{chapter}{Referencias Bibliogr�ficas}

\bibitem{LibroJoomla}Serrats, C., ``Programaci� de components MVC
per Joomla! 1.5.x'', CESI Inform�tica i Comunicacions, Girona, pp.
5 (2008). 
\end{singlespace}

\bibitem{MVCCakePHP}Cake Software Foundation, Inc. 2010. Understanding
Model-View-Controller. Disponible en Internet: http://book.cakephp.org/1.3/en/view/890/Understanding-Model-View-Controller,
consultado el 11 de diciembre de 2011.

\begin{singlespace}
\bibitem{ManualPHP}The PHP Group. 2011. Manual de PHP. Disponible
en Internet: http://www.php.net/manual/es/preface.php, consultado
el 10 de diciembre de 2011.

\bibitem{QueEsCakePHP}Cake Software Foundation, Inc. 2010. �Qu� es
CakePHP y por qu� hay que utilizarlo?. Disponible en Internet: http://book.cakephp.org/1.3/es/view/880/Qu\%C3\%A9-es-CakePHP-y-por-qu\%C3\%A9-hay-que-utilizarlo,
consultado el 10 de diciembre de 2011.

\bibitem{EstructuraCakePHP}Cake Software Foundation, Inc. 2010. Estructura
de CakePHP. Disponible en Internet: http://book.cakephp.org/es/view/893/Estructura-de-CakePHP,
consultado el 12 de diciembre de 2011.
\end{singlespace}

\bibitem{CakePHP-Request}Cake Software Foundation, Inc. 2010. A Typical
CakePHP Request. Disponible en Internet: http://book.cakephp.org/1.3/en/view/898/A-Typical-CakePHP-Request,
consultado el 12 de diciembre de 2011.

\begin{singlespace}
\bibitem{IntroAjax}P�rez, J. E., ``Introducci�n a AJAX'', librosweb.es,
pp. 6 (2008).

\selectlanguage{english}%
\bibitem{Ajax}\foreignlanguage{spanish}{Garret, Jesse. 2005. Ajax:
A New Approach to Web Applications. Disponible en Internet: http://www.adaptivepath.com/ideas/ajax-new-approach-web-applications,
consultado el 15 de diciembre de 2011.}

\selectlanguage{spanish}%
\bibitem{AUP}Ambler, Scott. 2009. The Agile Unified Process (AUP).
Disponible en Internet: http://www.ambysoft.com/unifiedprocess/agileUP.html,
consultado el 21 de diciembre de 2011.
\end{singlespace}

\bibitem{MYSQL}Oracle Corporation. 2011. Why MySQL?. Disponible en
Internet: http://www.mysql.com/why-mysql/, consultado el 2 de enero
de 2012.

\bibitem{PrincipiosCake}Cake Software Foundation, Inc. 2010. Principios
b�sicos de CakePHP. Disponible en Internet: http://book.cakephp.org/1.3/es/view/892/Principios-b\%C3\%A1sicos-de-CakePHP,
consultado el 2 de enero de 2012.

\bibitem{jQuery}The jQuery Project. 2010. jQuery: write less, do
more. Disponible en Internet: http://jquery.com/, consultado el 10
de octubre de 2011.

\bibitem{Cake1.3Api}Cake Software Foundation, Inc. 2010. The API
1.3.X. Disponible en Internet: http://api13.cakephp.org/ , consultado
el 3 de septiembre de 2011.

\bibitem{Cookbook}Cake Software Foundation, Inc. 2010. El manual
. Disponible en Internet: http://book.cakephp.org/1.3/es/view/876/El-manual,
consultado el 3 de septiembre de 2011.

\bibitem{build_Acl}Cake Software Foundation, Inc. 2010. An Automated
tool for creating ACOs. Disponible en Internet: http://book.cakephp.org/1.3/en/view/1549/An-Automated-tool-for-creating-ACOs,
consultado el 5 de septiembre de 2011.\end{thebibliography}

